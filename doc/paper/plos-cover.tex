\documentclass{letter}
\usepackage[margin=1in]{geometry}

\signature{William Stafford Noble \\
Associate Professor\\
Department of Genome Sciences\\
Department of Computer Science and Engineering\\
University of Washington}

\begin{document}

\begin{letter}{}

\opening{Dear editor:}

We are submitting a manuscript entitled ``Rapid and accurate peptide
identification from tandem mass spectra.''  The manuscript addresses
the general problem of identifying peptides from tandem mass spectra
via database search.  The manuscript is somewhat unusual because it
has several complementary aims.  First, we describe the first publicly
available implementation of the score functions that lie at the core
of {\sc Sequest}.  Second, we describe several algorithmic
improvements that significantly enhance our ability to accurately
identify peptides and to assign statistical significance scores to the
resulting identifications.  Several of these enhancements are novel,
whereas others have been reported in previous publications from our
group.  The software package, Crux, which implements all of the
methods described here is available with source code for
non-commercial users.

In addition to its scientific significance, this work has significant
intellectual property implications.  The {\sc Sequest} algorithm is
covered by a very broad patent, owned by the University of Washington,
that covers essentially all database search algorithms for peptide
identification from tandem mass spectra.  This patent is exclusively
licensed to Thermo Scientific (Waltham, MA), who markets {\sc Sequest}
along with their mass spectometry hardware.  In my opinion, the {\sc
Sequest} patent has been a considerable hindrance to algorithmic
development in this field, because according to the license agreement,
Thermo Scientific is obligated to pursue violators of the patent.
Indeed, I have been told that several discussions with violators of
the patent are currently underway.

{\sc Sequest} was patented and licensed over a decade ago, and now
times have changed.  The University of Washington would almost
certainly not issue an exclusive license today.  We are therefore in
negotiations with Thermo Scientific to allow non-commercial
distribution of source code that makes use of the {\sc Sequest}
patent.  My technology transfer office has assured me that this change
will happen in time for the publication of this manuscript.  I will be
happy to supply a letter from them that to that effect, if necessary.

%Please note that, although Crux uses the {\sc Sequest} patent, it does
%not violate {\sc Sequest}'s copyright.  I have a copy of the {\sc
%Sequest} source code, obtained from our technology transfer office,
%but the people in my lab who wrote the Crux code were not given access
%to that code.  Instead, I made available to them a compiled binary
%that prints out intermediate values, so as to allow them to more
%easily reverse engineer the code.

I am submitting, as supplementary material, a copy of the submitted
manuscript describing the $p$~value computation, mentioned in
Section~2.4.2.  If the reviewers would like to see a copy of the Crux
code, I could provide that as well.

\closing{Sincerely,}

\end{letter}
\end{document}
