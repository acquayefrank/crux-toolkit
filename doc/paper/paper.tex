\documentclass{bioinfo}
\copyrightyear{0000}
\pubyear{0000}

\begin{document}
\firstpage{1}

\title[Open-source peptide identification algorithm]{Open-source
peptide identification algorithm}
\author[Park \textit{et~al}]{Christopher Park$^{\rm a}$, Aaron Klammer$^{\rm a}$,
Lukas K\"{a}ll$^{\rm a}$, 
William S. Noble\,$^{\rm a,b,}$\footnote{To whom correspondence should be addressed}}
\address{
$^{\rm a}$Department of Computer Science and Engineering,
$^{\rm b}$Department of Genome Sciences, University of Washington,
  Seattle, WA, USA
}
\maketitle

\begin{abstract}
\section{Summary:}

We identify peptides better than anyone else! 

\section{Availability:}
\href{http://noble.gs.washington.edu/proj/crux}{http://noble.gs.washington.edu/proj/crux}
\section{Contact:} \href{noble@gs.washington.edu}{noble@gs.washington.edu}
\end{abstract}

\section{Introduction}

% Paragraph - Motivation   - Define the peptide identification problem
Peptide identification is hard!

Identifying proteins that are expressed in different cells under
varying conditions is one of the major challenges today in
Biology. With the recent instrumental and computational development,
Tandem mass spectrometry has become the method of choice for many
protein identification studies. However, protein identification
through spectral analysis has been technically challenging. Many
factors such as poor quality of spectra and large possible protein
candidates has contributed to the difficulty in protein
identification. In addition, the process is further complicated
because each spectrum is a result of a cleaved peptide product, while
the information we desire is the original source protein. Thus, the
protein identification problem can be divided into two steps. First,
the user must identify the peptide from the spectrum. Second, with the
peptide spectrum match information the source proteins must be
inferred. Majority of the effort in the field has been focused on
improving peptide identification, because still many algorithms fail
to accurately identify peptides in a timely manner. Our focus will be
concentrated on the peptide identification step.

% Paragraph - Related Work - SEQUEST, Tandem as open source?
SEQUEST\citep{eng:approach} There are two major approaches for peptide
identifications. First, the de novo sequencing approach, which tries
to infer the most statistically significant sequence from the spectrum
without any peptide database. Among such algorithms, include PepNovo
and Lutefisk. Second, the peptide database approach, where all
candidate peptides are compared to the query spectrum in which the
highest scoring peptide is identified. SEQUEST, InsPecT and
PeptideProphet are among the most wide used database search
algorithms. In practice, the peptide database approach is most widely
used because of the lack of accuracy in de novo sequencing.

% Paragraph - Related Work - Indexing
In algorithms that are used to search tandem MS spectra data against a
sequence database, a major bottleneck is the identification of
candidate peptides within a given mass window. Because scoring
functions are expensive procedure, all database algorithms produce a
smaller list of candidate peptides to be analyzed. However, generating
candidate peptides separately for each search can be wasteful for
large proteomes, because many times there are overlapping mass windows
of peptides. Therefore, by using a precomputed database, database
scoring algorithms can avoid re-generating peptides for each
search. Thus, resulting in dramatic reduction in runtime for large
proteome searches. TurboSEQUEST, a extension of SEQUEST, utilizes a
precomputed peptide database, however the application is not publicly
available.

% Paragraph - Related Work - Target decoy searches
% Paragraph - Related Work - Percolator
% Paragraph - Contribution - Four points of improvement
%  - Indexing
%  - "Open source" implementation of scoring functions
%  - Null peptide generation
%  - Percolator post processing
% Paragraph - Contribution - Brief results description


\begin{figure}
  \centering
  Insert figure here
  \caption{Peptide indexing allows rapid generation of candidate peptide
  lists.}
  \label{figure:indexing}
\end{figure}

\begin{figure}
  \centering
  \begin{tabular}{c}
  Insert figure here \\
  Insert figure here \\
  \caption{Re-implementation of Sp and Xcorr scoring functions.}
  \label{figure:indexing}
  \end{tabular}
\end{figure}

\begin{figure}
  \centering
  Insert figure here
  \caption{Positives vs. q-value (a measure of false discovery rate) for
  standard database search algorithms, and our implementation.}
  \label{figure:indexing}
\end{figure}



\section{Algorithms}

\subsection*{Indexing}
When given a protein fasta file as input, we generate an index of all
of its peptides on disk. The database created on disk is then used
during runtime as an input to generate candidate peptides within the
mass window of the query spectrum m/z. The main difficulties in
creating a database is first, dealing with memory shortage problems
caused by the large number of peptides,second, reducing the disk size
of the database. The key strategy to resolve the memory shortage
problem was to temporary store peptides in separate bins(file handles)
divided by mass interval, then sort each bin individually. By
partitioning the peptides into different mass bins, the total amount
of peptides needed to be sorted at one instance reduces dramatically
compared to sorting all the peptides together, thus reducing the
amount of memory needed to store the peptides. Reducing the disk size
of the database was accomplished by only storing information that was
not already present in the fasta file. For instance, instead of
storing the sequence of each individual peptide, only the protein
index, the start index and the peptide length are stored, thus when
needed, the program can quickly retrieve the sequence from the
original fasta file.

\subsection*{Scoring}
We re-implemented the two scoring functions Sp, Xcorr used in SEQUEST
for Crux.  First, the preliminary scoring function Sp is used to
extract the top 500 candidate peptides. To avoid erroneous scores by
noise, the spectrum is preprocessed by square rooting and smoothing
all peak intensities before extracting the top 200 abundant peaks to
be scored for Sp.  Second, The main scoring function Xcorr is used for
the final cross-correlation analysis. A similar preprocessing step
used in Sp is applied in Xcorr. For further detail on scoring function
and preprocessing steps refer to source code.

\subsection*{Null peptide generation}
\subsection*{Percolator}
\subsection*{Parallelization}

\section{Results}

Hardware: All experiments were performed on RedHat linux machine with
Athlon MP 2000, 2 x 1.67 GHz processor and 2GB of RAM.  First, using
Crux, we precomputed a peptide database for tryptic peptides of length
6 to 50 amino acids while allowing missed cleavage sites in the
nr-db(06/02/2007, 3,292,818 proteins). This procedure required 19hrs
and 14GB of disk space.

Second, we verified that our re-implemented SP and Xcorr scoring
functions are identical to the scores computed by SEQUEST.

Third, to test our indexing method for candidate peptide generation,
we compared its runtime to SEQUEST.  We use a modified version of
SEQUEST that only computes Sp (and not XCorr), and we compare its
running time to that of Crux.  For both methods, we performed searches
with 100 different spectra, using a mass tolerance of 3.  SEQUEST
required 3.55hrs clock time. Crux required 0.58hrs clock time, which
is roughly a six-fold speed improvement. We also tested Crux using a
smaller mass tolerance window, such as might be used with
high-resolution mass spectrometry instrumentation.  As the mass
tolerance window decreases, Crux's speed improves, because fewer
candidate peptides are retrieved from the database.  SEQUEST, by
contrast, must traverse the entire database, regardless of the mass
tolerance.  With a mass tolerance window of 0.1, SEQUEST requires
3.5hrs clock time, whereas Crux requires 0.1hrs.  This represents an
30-fold improvement.These results show that our method of using a
pre-computed peptide database outperforms SEQUEST, especially on large
proteomes or when searching with a small mass tolerance.

Finally, we test our false discovery rate for Crux using the
Percolator algorithm compared to stadndard database searching
algorithms.

\subsection*{Datasets}
We use the following publicly available tandem mass spectrometry data
sets.

\section{Discussion}
% Mention Ting Chen's contribution
Discuss this!

\section*{Acknowledgment}

Acknowledge this!

% We thank Grant and Tobias Mann for helpful advice. Funding:
% NIH grants U01~HG003161 and R01~GM071923.


% N.B. To use the bibliography, check out the 'refs' CVS module and add it to
% your Bibtex path.

\bibliographystyle{plainnat}
\bibliography{refs} 

\end{document}
