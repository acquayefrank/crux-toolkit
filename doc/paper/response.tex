\documentclass{article}
\usepackage[margin=1in]{geometry}
\usepackage[dvips]{color}
\usepackage{url}

%%
%% You could use latex or pdflatex to process this file, since there
%% are no graphics, BUT ...
%%
%% To get the \textcolor command to display, I had to use regular
%% latex.  The dvi file does not show the red text, but after
%% converting to either postscript (using dvips) or pdf (using
%% dvipdf), the color shows.
%%

\newcommand{\breview}{\begin{quotation}\begin{em}\noindent}
\newcommand{\ereview}{\end{em}\end{quotation}}

\begin{document}

\hspace*{3.0in}\today

\vspace*{3ex}

\noindent
Dear editor:

\vspace*{1ex}

We are submitting a revised version of the manuscript ``Unsupervised
segmentation of genomic data,'' in which we attempt to address all of
the reviewers' concerns.  Our detailed responses are given below.  In
both this letter and in the main text, we have highlighted new changes
to the main text in red.

We hope that the revised version sufficiently addresses the reviewers'
comments, and we look forward to seeing it published in {\em
Bioinformatics}.

\vspace*{1ex}

\noindent
Sincerely,

\hspace*{1ex}

\noindent
Robert Thurman\\
Division of Medical Genetics\\
Department of Genome Sciences\\
University of Washington


\section*{Editor's comment}

\breview You should include some evaluation of the method within
supplementary materials.  \ereview

We have performed validation and accuracy analysis and posted the
results to the HMMSeg website
(\url{http://noble.gs.washington.edu/proj/hmmseg}).  We have added the
following to the text.

\begin{quotation}
HMMSeg is implemented in Java for platform independence.  It has been
successfully tested on Windows and UNIX-style
systems. \textcolor{red}{Validation and accuracy test results are
available on the website.}
\end{quotation}

\section*{Reviewer 1 comments}

\breview
The HMMs are based on Gaussian distributions with no covariance
modeling. This should be stressed more and has some impacts; i.e.,
there is no modeling of dependency between the input data. Also, the
last sentence of the abstract is therefore misleading -- there is no
option to (directly) integrate sequence data, which would be based on
a discrete alphabet; all input data needs to be continuous.
\ereview

We changed the paragraph on Hidden Markov Models to read:

\begin{quotation}
HMMSeg uses Gaussian emission distributions\textcolor{red}{, with
diagonal covariance for multiple datasets (assuming independence
between variables}), and supports both the Viterbi and posterior
decoding methods for state assignments.
\end{quotation}

Also, we changed ``sequence'' to ``expression'' in the abstract.

\breview I know the space is severely limited, but another sentence or
two on wavelets would be good; also, the choice of the scale parameter
is not clear: is it completely heuristic? wouldn't it make sense to
have different wavelets for different input data -- figure 2
(e.g. cell cycle vs histones) seems to indicate this -- and is this
possible?
\ereview

In Figure 2, all datasets are smoothed to a common 64kb scale.
However, since the datasets started at different resolutions, some
were smoothed more than others to reach that scale.  So in this sense,
it does make sense to treat and smooth individual datasets separately,
and this can be done by using HMMSeg as a preprocessing step (without
segmenting, as indicated in the text).  The website gives instructions
for doing this, in describing the steps required for generating the
segmentation in Figure 2.  As for more on wavelets and scale, we have
linked a new supplementary document on wavelets to the webpage, and
added the following to the paragraph on {\em Wavelet Smoothing} in the
manuscript:

\begin{quotation}
HMMSeg uses the LA(8) family of wavelets for all wavelet transforms.
\textcolor{red}{The choice of appropriate wavelet scale is
application-dependent, and can be informed by prior biological
information about the scale of features of interest, or by
trial-and-error to achieve, say, a desired segment length
distribution. See the website for further details on wavelet
smoothing.}\end{quotation}

\section*{Reviewer 2 comments}

\subsection*{1) Comments to the Author}

\breview  The presented tool proposes an unsupervised method for
functional segmentation of multivariate continuous genomic data.  The
manuscript is written with clarity and exposes precisely the
specifities of the tool.

The interest of wavelet smoothing, as it permits to clean on small
scale experimental artifacts, is justified. It also corresponds to a
new application of wavelet methods to genomic data.

Nevertheless, after wavelet smoothing, the advantages of an HMM
segmentation over simple thresholding is still unclear and could be
emphasized in the Example section (see also point 2a).  
\ereview

We added the following to the Example section.

\begin{quotation}
Examples of large-scale domains delineated by HMMSeg are pictured in
Figure~2 \textcolor{red}{(see website for details).  Here we see the
advantages of using HMMs over simple thresholding techniques, the
logic of which breaks down in scenarios with multiple datasets and few
states.}
\end{quotation}

\subsection*{2a) Major Comments}

\breview
In order to guide the first time user and to highlight the interest
of the method, an assessment of software accuracy for different
parameter settings should be given. This could be a simple sentence in
the article completed with a more detailled analysis on the webpage.
\ereview

See our response to the Editor's comment, above.
 
\breview More details should be given concerning wavelet smoothing, by
a short example in the article or by a detailled example on the
software webpage.  More generally the software webpage should be
improved, with at least one general example.  \ereview

Based on this and other reviewer comments, we have expanded the
webpage considerably.  We have added a link to an example analysis,
describing how the segmentation pictured in Figure 2 was arrived at,
including step-by-step instructions starting from the raw data, and
referred to the website in the main text. 

\subsection*{2b) Minor Comments}

\breview I was unable to run the program due to a version
incompatibility (the jar file was build for JVM version 1.5 and
higher). It would be surprising that HMMseg would need such an high
version and I would suggest to build it in order to allow maximum
compatibility with older versions.  \ereview

We have built HMMSeg using java version 1.5, which was released over
two years ago and has superior garbage collection to earlier versions.
While we are loth to keep potential users from using our software
because of version incompatibility, we do not have the resources to
install and maintain multiple versions of the java compiler and/or our
jar file at this time.

\breview
Getting the datasets used for the experiment was not obvious. I would
suggest to put example datafiles on the authors website.
\ereview

We have included an example analysis on the website, which includes
instructions for downloading the raw data and instructions for
generating, and links to, preprocessed data.

\breview
On l. 54-56, col. 1, p. 1, the authors didn't mention the ghmm tool
(http://ghmm.org/) that allows general modelisation using Hidden
Markov Models, and, to the exception of wavelet smoothing, can do the
same job as HMMseg, using more precise models.
\ereview

We amended the relevant paragraph as follows:

\begin{quotation}
As a platform for segmenting a wide variety of genomic data, HMMSeg is
distinguished from existing programs using HMMs, which typically fall
under two categories: toolboxes for applications in any field, such as
htk [Young et al., 1995] \textcolor{red}{and GHMM
(\url{http://ghmm.org})}; or biological application-specific tools
that use HMMs, such as \ldots
\end{quotation}

\end{document}
