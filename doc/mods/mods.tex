\documentclass[12pt]{article}
\usepackage[margin=1in]{geometry}
\usepackage{algpseudocode}
\usepackage{algorithm}

\begin{document}


\begin{algorithm}
\caption{{\bf Searching with Modifications} Inputs: (1) a charged
  spectrum S containing a mass and a list of peaks, (2) a set Q of
  sequences each containing a mass and an ordered list of amino acids,
  (3) a set M of peptide modifications, (4) the number N of PSMs to
  output per spectrum. Outputs: a set R of PSMs.  Each PSM contains a
  peptide sequence (q) and two scores (x, sp).
  \label{algorithm:search}}

\begin{algorithmic}[1]
\Procedure{SearchOneSpectrum}{S, Q, M, N}

\State R $\gets$ []
\Comment Initialize an empty set of results.

\For{i $\gets$ 1 ... $|$M$|$}
   \State Qs $\gets$ selectPeptidesInMassRange(S.mass, M[i], Q)
   \Comment Algorithm~\ref{algorithm:select}
   
   \For{j $\gets$ 1 ... $|$Qs$|$}
     \State r $\gets$ newPsm(Qs[j])
     \Comment Create new PSM and set sequence.
     \State r.sp $\gets$ scorePeptideSp(r.q, S.peaks)
     \Comment Set Sp score for PSM.
     \State R += r
     \Comment Add PSM to set of results.
   \EndFor

   \State R $\gets$ sortBySp(R)
   \State R $\gets$ truncateTopN(R, 500)
   \Comment Keep only best-scoring PSMs for Xcorr scoring.

   \For{j $\gets$ 1 ... $|$R$|$}
     \State R[j].x $\gets$ scorePeptideXcorr(R[j].q, S.peaks)
   \EndFor

   \State R $\gets$ sortByXcorr(R)
   \State R $\gets$ truncateTopN(R, N)
   \Comment Keep only N of the best PSMs.

\EndFor

\State \Return{R}
\EndProcedure
\end{algorithmic}
\end{algorithm}

\begin{algorithm}
\caption{ Inputs: (1) the target mass (m), (2) a peptide modification (M)
  containing a mass shift (mass) and a list of amino acid
  modifications to be applied (aas), and (3) a set Q of possible sequence.
  Output: a set of modified peptides in the specified mass range.
  \label{algorithm:select}}

\begin{algorithmic}[1]
\Procedure{selectPeptidesInMassRange}{m, M, Q}
  \State Qu $\gets$ selectUnmodifiedPeptides(m - M.mass, Q)
   
  \State Qm $\gets$ []
  \Comment Initialize set of modified peptides.

  \For{i $\gets$ 1 ... $|$Qu$|$}
    \If{isModifiable(Qu[i], M.aas)}
      \State Qm += generateAllModifiedForms(Qu[i], M.aas)
      \Comment Algorithm~\ref{algorithm:generate}
    \EndIf
  \EndFor

  \State \Return{Qm}
\EndProcedure
\end{algorithmic}
\end{algorithm}

\begin{algorithm}
\caption{{\bf Modifying a Peptide Sequence} Inputs: (1) a sequence q
  containing an ordered list of amino acids, (2) a list M of amino acid
  modifications.  Each modification M[i] consists three parts: a
  unique mass shift (M[i].delta), the maximum number of times that
  shift will be applied to the sequence (M[i].count), and a set of
  amino acids to which this mass shift can be applied (M[i].aas).
  Output: a set containing many copies of the given peptide,
  corresponding to all the ways that the given modification can be
  applied to the peptide.
  \label{algorithm:generate}}

\begin{algorithmic}[1]
\Procedure{generateAllModifiedForms}{q, M}

  \State Q $\gets$ [q]
  \Comment Initialize list with the unmodified sequence.

  \For{i $\gets$ 1 ... $|$M$|$}
  \Comment Apply each modification to the list of sequences.
    \For{i $\gets$ 1 ... M[i].count}
    \Comment Apply modification specified number of times 
      \State n $\gets |$Q$|$
      \For{j $\gets$ 0 ... n - 1}
      \Comment Apply to each sequence
        \State Q += applyModToSequence(M[i].delta, M[i].aas, Q[j], j)
        \Comment Algorithm~\ref{algorithm:apply}
      \EndFor
    \EndFor
  \EndFor

  \State \Return{Q}

\EndProcedure
\end{algorithmic}
\end{algorithm}

\begin{algorithm}
\caption{{\bf Modifying one Residue of a Peptide Sequence} Inputs: (1)
  the mass shift m to apply, (2) the set A of amino acids to which the
  shift can be applied, (3) a peptide sequence q and (4) the number n
  of residues in the sequence to which this modification has already
  been applied.  Output: a set contraining copies of the given
  peptide, corresponding to all ways that the given modification can
  be applied to a single residue of the peptide.
  \label{algorithm:apply}} 
\begin{algorithmic}[1]
\Procedure{applyModToSequence}{m, A, q, n}

  \State Q $\gets$ []
  \Comment Initialize list of modified sequences.

  \For{i $\gets$ 1 ... $|$q$|$}
  \Comment Skip over the first n residues already modified.
    \If{ isModified(q[i], m) }
      \State n $\gets$ n - 1
    \EndIf
    \If{ n == 0 }
      \State startIdx $\gets$ i
      \State break
    \EndIf
  \EndFor

  \For{j $\gets$ startIdx ... $|$q$|$}
  \Comment Try to modify each of the remaining residues
    \If{ q[j] $\in$ A }
    \Comment If the residue can be modified.
      \State Q += createModifiedCopy(q, j, m)
      \Comment Create a copy of q and modify q[j].
    \EndIf
  \EndFor

  \State \Return {Q}

\EndProcedure
\end{algorithmic}
\end{algorithm}



\end{document}
