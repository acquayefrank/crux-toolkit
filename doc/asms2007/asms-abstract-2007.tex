\documentclass[12pt]{article}

\usepackage[margin=1in]{geometry}

\begin{document}

\title{Improved candidate peptide generation using a precomputed
peptide database}

\author{
Christopher Park\\
Department of Genome Sciences\\
University of Washington\\
Seattle, WA, USA
\and
Aaron Klammer\\
Department of Genome Sciences\\
University of Washington\\
Seattle, WA, USA
\and
William Stafford Noble\\
Department of Genome Sciences\\
Department of Computer Science and Engineering\\
University of Washington\\
Seattle, WA, USA
}

\maketitle

\section{Introduction}

In algorithms that are used to search
tandem-MS-spectra data against a sequence database, a major bottle-neck is
the identification of candidate peptides within a given mass window. We
present an efficient method for generating peptides in a particular mass
window by using a pre-computed peptide database. Our method reduces the
runtime dramatically for large proteome searches compared to SEQUEST
(Eng et al. 1995), a common identification algorithm that does not use
any pre-computed database. We provide our indexing code as a tool for
the wider proteomics community. While there are other recent attempts
to avoid re-generating peptides for each search (Dutta et al, 2007,
[turbo SEQUEST reference]), these programs are either not publicly
available or require large amounts of disk space.

\section{Methods}

 First a user creates a peptide database on
disk for the proteome of interest. Instead of storing the sequence of
each individual peptide, done in methods like TurboSequest [turbo SEQUEST
reference], our method only stores the protein index, the start index
and the peptide length. This is a key aspect of our method which allows
quick retrieval of peptide sequences from the original Fasta file, while
also maintaining a manageable peptide database size.  Once the peptide
database is on disk, for all queries given a mass window, peptides
that meet the constraint are simply parsed out from the pre-computed
peptide database. This use of the peptide database eliminates the need
to re-compute peptides for each search.

\section{Preliminary Data}
We tested our method by comparing
its runtime to SEQUEST. First, the runtime comparison was conducted by
running 100 different spectra against the MSDB proteome(2344227 proteins) using both
SEQUEST and our method with a mass tolerance of 3, for tryptic cleavage
peptides. SEQUEST uses a preliminary scoring method called `Sp' to
extract the top 500 candidate peptides before the final cross-correlation
analysis. Since we are unable to directly measure the runtime for SEQUEST
to generate its candidate peptides, we compare the combined runtime
of generating a list of candidate peptides, and performing Sp on all
the peptides. We have implemented Sp in our method that is identical
to SEQUEST. The results show that our method of using a pre-computed
peptide database outperforms SEQUEST, especially on large proteomes. The
SEQUEST runtime against the MSDB proteome took 213mins16sec clock time
and 212mins 20sec CPU time. Our method took 35mins 36sec clock time and
34mins 26sec CPU time, which is roughly a 6 fold speed improvement.
We also tested our results on smaller mass tolerance windows. With
high-resolution mass spectrometry instrumentation, such as the FT-ICR
LTQ, a smaller mass tolerance window can be applied to reduce the number
of candidate peptides. We repeated the same experiments described
above but instead used a smaller mass tolerance window of 0.1. The
SEQUEST runtime against the MSDB proteome took XXmins XXsec clock time,
compared to XXmins XXsec for our method. In SEQUEST, all peptides must
be evaluated on each separate search, thus there is a minimal runtime
reduction for smaller mass windows. However, in our method the runtime is
directly proportional to the number of peptides generated. Therefore, for
smaller mass tolerances windows the runtime can be dramatically reduced.
Finally, the tryptic cleavage peptide database creation required 16hrs
of computation time and 14GB disk size for the MSDB proteome. [Maybe a
comparison to TurboSEQUEST].

\section{Novel Aspect}
When compared to current searching methods,
our method uses a compact pre-computed peptide database that yields
faster peptide searches for large proteomes.

\end{document}

