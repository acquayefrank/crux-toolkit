\documentclass[12pt]{article}

\usepackage[margin=1in]{geometry}

\begin{document}

\title{Improved candidate peptide generation using a precomputed
peptide database}

\author{
Christopher Park\\
Department of Genome Sciences\\
University of Washington\\
Seattle, WA, USA
\and
Aaron Klammer\\
Department of Genome Sciences\\
University of Washington\\
Seattle, WA, USA
\and
William Stafford Noble\\
Department of Genome Sciences\\
Department of Computer Science and Engineering\\
University of Washington\\
Seattle, WA, USA
}

\maketitle

\section{Introduction}

In algorithms that are used to search tandem MS spectra data against a
sequence database, a major bottleneck is the identification of
candidate peptides within a given mass window. We present an efficient
method for generating peptides in a particular mass window by using a
pre-computed peptide database. Our method reduces the runtime
dramatically for large proteome searches compared to SEQUEST (Eng et
al. 1995).  The peptide index method is implemented in a publicly available software
package called ``Crux.''  Other methods that avoid re-generating
peptides for each search are either not publicly available (Dutta et
al, 2007) or require large amounts of disk space like TurboSequest.

\section{Methods}

Searching with Crux requires two phases.  First, a user creates a
peptide database on disk for the proteome of interest.  Instead of
storing the sequence of individual peptides, as in TurboSequest, 
Crux only stores the protein index, the start
index and the peptide length.  This allows quick retrieval of peptide
sequences from the database, while maintaining a manageable peptide
database size.  Once the peptide database is on disk, retrieving all
peptides within a given mass window can be accomplished efficiently.
This use of the peptide database eliminates the need to re-scan the
sequence file for each search.

\section{Preliminary Data}

SEQUEST uses a preliminary score called ``Sp'' to extract the top 500
candidate peptides prior to the final cross-correlation analysis.
We re-implemented Sp in Crux, and verified that it is identical to the
Sp computed by SEQUEST.

% Use either clock time or CPU time, not both.
% Report all times in a single unit, with decimals (e.g., 1.23 hrs
% or 82.4 min).

First, using Crux, we precomputed a peptide database for the tryptic
peptides in the MSDB (2344227 proteins).  This procedure required 16
hrs and 14GB of disk space.

We then tested our method by comparing its runtime to SEQUEST.  We use
a modified version of SEQUEST that only computes Sp (and not XCorr),
and we compare its running time to that of Crux.  For both methods, we
performed searches with 100 different spectra, using a mass tolerance
of 3.  SEQUEST required 3.55hrs clock time. Crux required 0.58hrs clock time, 
which is roughly a six-fold speed improvement.

We also tested Crux using a smaller mass tolerance window, such
as might be used with high-resolution mass spectrometry
instrumentation.  As the mass tolerance window decreases, Crux's speed
improves, because fewer candidate peptides are retrieved from the
database.  SEQUEST, by contrast, must traverse the entire database,
regardless of the mass tolerance.  With a mass tolerance window of
0.1, SEQUEST requires 3.5hrs clock time, whereas Crux requires 
0.1hrs.  This represents an 30-fold improvement.

These results show that our method of using a pre-computed peptide
database outperforms SEQUEST, especially on large proteomes or when
searching with a small mass tolerance.

\section{Novel Aspect}

When compared to current searching methods, our method uses a compact
pre-computed peptide database that yields faster peptide searches for
large proteomes.

\end{document}

